\chapter{Практическая часть}

% \section{Особенности программной реализации}

\section{Цель и задачи эксперимента}
Целью работы является исследование эффективности методов кластеризации Гат-Гевы, K-средних и C-средних на размеченном наборе текстовых данных. Для этого необходимо:
\begin{enumerate}
    \item Предобработать тексты и создать векторные представления документов двумя методами;
    \item Провести кластеризацию с использованием указанных методов, варьируя количество кластеров \( k \);
    \item Рассчитать метрики внутрикластерного и межкластерного расстояний для оценки качества разбиения;
    \item Сравнить полученные кластеризации с экспертной разметкой.
\end{enumerate}

\section{Описание набора данных}
Работа проводились на предоставленном размеченном наборе текстовых данных. Набор содержит \( N = 50 \) текстовых документов, относящихся к \(K = 7 \) тематическим кластерам согласно экспертной разметке.

\section{Предобработка текста и векторизация}
Векторизация документов выполнялась в два этапа: предобработка текста и создание числового вектора.

В ходе первого этапа исходный текст очищался от знаков препинания, символов перевода строк и приводился к нижнему регистру. После этого каждый документ разбивался на отдельные слова (токены) с помощью простого разделителя по пробелам. 

Для получения начальных форм слов использовалась библиотека \texttt{pymorphy3}. Каждый токен, полученный на предыдущем этапе, преобразовывался в свою нормальную форму (лемму).

\section{Создание векторных представлений}
После лемматизации для каждого из двух методов (мешок словоформ и мешок лемм) выполнялась векторизация. В результате каждый документ \( d_i \) был представлен как вектор \( \mathbf{x}_i \), где каждая компонента соответствовала весу конкретного слова (словоформы или леммы) в документе. Служебные части речи при векторизации не учитывались.

\section{Результаты}





% \includelistingpretty
%     {out.txt} % Имя файла с расширением (файл должен быть расположен в директории inc/lst/)
%     {} % Язык программирования (необязательный аргумент)
%     {Пример работы с консольной программой} % Подпись листинга



\chapter*{ЗАКЛЮЧЕНИЕ}


