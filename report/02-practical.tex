\chapter{Практическая часть}

% \section{Особенности программной реализации}

\section{Цель и задачи эксперимента}
Целью работы является исследование эффективности методов кластеризации Гат-Гевы, K-средних и C-средних на размеченном наборе текстовых данных. Для этого необходимо:
\begin{enumerate}
    \item Предобработать тексты и создать векторные представления документов двумя методами;
    \item Провести кластеризацию с использованием указанных методов, варьируя количество кластеров \( k \);
    \item Рассчитать метрики внутрикластерного и межкластерного расстояний для оценки качества разбиения;
    \item Сравнить полученные кластеризации с экспертной разметкой.
\end{enumerate}

\section{Описание набора данных}
Работа проводились на предоставленном размеченном наборе текстовых данных. Набор содержит \( N = 50 \) текстовых документов, относящихся к \(K = 7 \) тематическим кластерам согласно экспертной разметке.

\section{Предобработка текста и векторизация}
Векторизация документов выполнялась в два этапа: предобработка текста и создание числового вектора.

В ходе первого этапа исходный текст очищался от знаков препинания, символов перевода строк и приводился к нижнему регистру. После этого каждый документ разбивался на отдельные слова (токены) с помощью простого разделителя по пробелам. 

Для получения начальных форм слов использовалась библиотека \texttt{pymorphy3}. Каждый токен, полученный на предыдущем этапе, преобразовывался в свою нормальную форму (лемму).

\section{Создание векторных представлений}
После лемматизации для каждого из двух методов (мешок словоформ и мешок лемм) выполнялась векторизация. В результате каждый документ \( d_i \) был представлен как вектор \( \mathbf{x}_i \), где каждая компонента соответствовала весу конкретного слова (словоформы или леммы) в документе. Служебные части речи при векторизации не учитывались.

\section{Результаты кластеризации с лемматизацией}

\subsection{Метод К-средних}

Ниже на рисунках \ref{img:k_7}-\ref{img:k_10} представлены результаты кластеризации методом К-средних с лемматизацией.

\includeimage
    {k_7} % Имя файла без расширения (файл должен быть расположен в директории inc/img/)
    {f} % Обтекание (без обтекания)
    {h} % Положение рисунка (см. figure из пакета float)
    {1\textwidth} % Ширина рисунка
    {Результат кластеризации методом К-средних, при К=7} % Подпись рисунка

\newpage

\includeimage
    {k_2} % Имя файла без расширения (файл должен быть расположен в директории inc/img/)
    {f} % Обтекание (без обтекания)
    {h} % Положение рисунка (см. figure из пакета float)
    {1\textwidth} % Ширина рисунка
    {Результат кластеризации методом К-средних, при К=2} % Подпись рисунка

\includeimage
    {k_3} % Имя файла без расширения (файл должен быть расположен в директории inc/img/)
    {f} % Обтекание (без обтекания)
    {h} % Положение рисунка (см. figure из пакета float)
    {1\textwidth} % Ширина рисунка
    {Результат кластеризации методом К-средних, при К=3} % Подпись рисунка

\newpage


\includeimage
    {k_4} % Имя файла без расширения (файл должен быть расположен в директории inc/img/)
    {f} % Обтекание (без обтекания)
    {h} % Положение рисунка (см. figure из пакета float)
    {1\textwidth} % Ширина рисунка
    {Результат кластеризации методом К-средних, при К=4} % Подпись рисунка

\includeimage
    {k_5} % Имя файла без расширения (файл должен быть расположен в директории inc/img/)
    {f} % Обтекание (без обтекания)
    {h} % Положение рисунка (см. figure из пакета float)
    {1\textwidth} % Ширина рисунка
    {Результат кластеризации методом К-средних, при К=5} % Подпись рисунка

\newpage


\includeimage
    {k_6} % Имя файла без расширения (файл должен быть расположен в директории inc/img/)
    {f} % Обтекание (без обтекания)
    {h} % Положение рисунка (см. figure из пакета float)
    {1\textwidth} % Ширина рисунка
    {Результат кластеризации методом К-средних, при К=6} % Подпись рисунка

\includeimage
    {k_10} % Имя файла без расширения (файл должен быть расположен в директории inc/img/)
    {f} % Обтекание (без обтекания)
    {h} % Положение рисунка (см. figure из пакета float)
    {1\textwidth} % Ширина рисунка
    {Результат кластеризации методом К-средних, при К=10} % Подпись рисунка

\newpage

\subsection{Метод С-средних}

Ниже на рисунках \ref{img:c_7}-\ref{img:c_10} представлены результаты кластеризации методом C-средних с лемматизацией.

\includeimage
    {c_7} % Имя файла без расширения (файл должен быть расположен в директории inc/img/)
    {f} % Обтекание (без обтекания)
    {h} % Положение рисунка (см. figure из пакета float)
    {1\textwidth} % Ширина рисунка
    {Результат кластеризации методом C-средних, при К=7} % Подпись рисунка

\includeimage
    {c_2} % Имя файла без расширения (файл должен быть расположен в директории inc/img/)
    {f} % Обтекание (без обтекания)
    {h} % Положение рисунка (см. figure из пакета float)
    {1\textwidth} % Ширина рисунка
    {Результат кластеризации методом C-средних, при К=2} % Подпись рисунка

\newpage


\includeimage
    {c_3} % Имя файла без расширения (файл должен быть расположен в директории inc/img/)
    {f} % Обтекание (без обтекания)
    {h} % Положение рисунка (см. figure из пакета float)
    {1\textwidth} % Ширина рисунка
    {Результат кластеризации методом C-средних, при К=3} % Подпись рисунка

\includeimage
    {c_4} % Имя файла без расширения (файл должен быть расположен в директории inc/img/)
    {f} % Обтекание (без обтекания)
    {h} % Положение рисунка (см. figure из пакета float)
    {1\textwidth} % Ширина рисунка
    {Результат кластеризации методом C-средних, при К=4} % Подпись рисунка

\newpage


\includeimage
    {c_5} % Имя файла без расширения (файл должен быть расположен в директории inc/img/)
    {f} % Обтекание (без обтекания)
    {h} % Положение рисунка (см. figure из пакета float)
    {1\textwidth} % Ширина рисунка
    {Результат кластеризации методом C-средних, при К=5} % Подпись рисунка

\includeimage
    {c_6} % Имя файла без расширения (файл должен быть расположен в директории inc/img/)
    {f} % Обтекание (без обтекания)
    {h} % Положение рисунка (см. figure из пакета float)
    {1\textwidth} % Ширина рисунка
    {Результат кластеризации методом C-средних, при К=6} % Подпись рисунка

\newpage


\includeimage
    {c_10} % Имя файла без расширения (файл должен быть расположен в директории inc/img/)
    {f} % Обтекание (без обтекания)
    {h} % Положение рисунка (см. figure из пакета float)
    {1\textwidth} % Ширина рисунка
    {Результат кластеризации методом C-средних, при К=10} % Подпись рисунка

\section{Результаты кластеризации без лемматизации}

\subsection{Метод К-средних}

Ниже на рисунках \ref{img:wk_7}-\ref{img:wk_10} представлены результаты кластеризации методом К-средних без лемматизации.

\includeimage
    {wk_7} % Имя файла без расширения (файл должен быть расположен в директории inc/img/)
    {f} % Обтекание (без обтекания)
    {h} % Положение рисунка (см. figure из пакета float)
    {1\textwidth} % Ширина рисунка
    {Результат кластеризации методом К-средних, при К=7} % Подпись рисунка

\newpage


\includeimage
    {wk_2} % Имя файла без расширения (файл должен быть расположен в директории inc/img/)
    {f} % Обтекание (без обтекания)
    {h} % Положение рисунка (см. figure из пакета float)
    {1\textwidth} % Ширина рисунка
    {Результат кластеризации методом К-средних, при К=2} % Подпись рисунка

\includeimage
    {wk_3} % Имя файла без расширения (файл должен быть расположен в директории inc/img/)
    {f} % Обтекание (без обтекания)
    {h} % Положение рисунка (см. figure из пакета float)
    {1\textwidth} % Ширина рисунка
    {Результат кластеризации методом К-средних, при К=3} % Подпись рисунка

\newpage


\includeimage
    {wk_4} % Имя файла без расширения (файл должен быть расположен в директории inc/img/)
    {f} % Обтекание (без обтекания)
    {h} % Положение рисунка (см. figure из пакета float)
    {1\textwidth} % Ширина рисунка
    {Результат кластеризации методом К-средних, при К=4} % Подпись рисунка

\includeimage
    {wk_5} % Имя файла без расширения (файл должен быть расположен в директории inc/img/)
    {f} % Обтекание (без обтекания)
    {h} % Положение рисунка (см. figure из пакета float)
    {1\textwidth} % Ширина рисунка
    {Результат кластеризации методом К-средних, при К=5} % Подпись рисунка

\newpage


\includeimage
    {wk_6} % Имя файла без расширения (файл должен быть расположен в директории inc/img/)
    {f} % Обтекание (без обтекания)
    {h} % Положение рисунка (см. figure из пакета float)
    {1\textwidth} % Ширина рисунка
    {Результат кластеризации методом К-средних, при К=6} % Подпись рисунка

\includeimage
    {wk_10} % Имя файла без расширения (файл должен быть расположен в директории inc/img/)
    {f} % Обтекание (без обтекания)
    {h} % Положение рисунка (см. figure из пакета float)
    {1\textwidth} % Ширина рисунка
    {Результат кластеризации методом К-средних, при К=10} % Подпись рисунка

\newpage

\subsection{Метод С-средних}

Ниже на рисунках \ref{img:wc_7}-\ref{img:wc_10} представлены результаты кластеризации методом C-средних без лемматизации.

\includeimage
    {wc_7} % Имя файла без расширения (файл должен быть расположен в директории inc/img/)
    {f} % Обтекание (без обтекания)
    {h} % Положение рисунка (см. figure из пакета float)
    {1\textwidth} % Ширина рисунка
    {Результат кластеризации методом C-средних, при К=7} % Подпись рисунка

\includeimage
    {wc_2} % Имя файла без расширения (файл должен быть расположен в директории inc/img/)
    {f} % Обтекание (без обтекания)
    {h} % Положение рисунка (см. figure из пакета float)
    {1\textwidth} % Ширина рисунка
    {Результат кластеризации методом C-средних, при К=2} %Подпись рисунка

\newpage


\includeimage
    {wc_3} % Имя файла без расширения (файл должен быть расположен в директории inc/img/)
    {f} % Обтекание (без обтекания)
    {h} % Положение рисунка (см. figure из пакета float)
    {1\textwidth} % Ширина рисунка
    {Результат кластеризации методом C-средних, при К=3} % Подпись рисунка

\includeimage
    {wc_4} % Имя файла без расширения (файл должен быть расположен в директории inc/img/)
    {f} % Обтекание (без обтекания)
    {h} % Положение рисунка (см. figure из пакета float)
    {1\textwidth} % Ширина рисунка
    {Результат кластеризации методом C-средних, при К=4} % Подпись рисунка

\newpage


\includeimage
    {wc_5} % Имя файла без расширения (файл должен быть расположен в директории inc/img/)
    {f} % Обтекание (без обтекания)
    {h} % Положение рисунка (см. figure из пакета float)
    {1\textwidth} % Ширина рисунка
    {Результат кластеризации методом C-средних, при К=5} % Подпись рисунка

\includeimage
    {wc_6} % Имя файла без расширения (файл должен быть расположен в директории inc/img/)
    {f} % Обтекание (без обтекания)
    {h} % Положение рисунка (см. figure из пакета float)
    {1\textwidth} % Ширина рисунка
    {Результат кластеризации методом C-средних, при К=6} % Подпись рисунка

\newpage


\includeimage
    {wc_10} % Имя файла без расширения (файл должен быть расположен в директории inc/img/)
    {f} % Обтекание (без обтекания)
    {h} % Положение рисунка (см. figure из пакета float)
    {1\textwidth} % Ширина рисунка
    {Результат кластеризации методом C-средних, при К=10} % Подпись рисунка

\section{Метод Гат-Гевы}

Решение задачи кластеризации методом Гат-Гевы выполнялась на языке \texttt{matlab} ввиду отсутствия готовых реализаций на языке \texttt{python} (справедливости ради, мною были найдены готовые реализации этого метода кластеризации, однако ввиду того что они были написаны достаточно давно, текущие версии библиотек их не поддерживали, а заниматься их реанимацией мне не хотелось, посему было принято решение реализовать этот метод в \texttt{matlab}, где Гат-Гева входит в состав одного из \textit{toolbox}).

В \texttt{matlab} метод Гат-Гевы реализован в рамках \texttt{Fuzzy Logic Toolbox}.Результат задачи кластеризации методом Гат-Гевы при стандартных размерностях векторов документов, представленный на рисунке \ref{img:g_7} не дают ожидаемого результата.

Однако, предварительно перед кластеризацией уменьшив размерность векторов методом главных компонент удалось получить результат представленный на рисунке \ref{img:g_7_pca}.

Ниже на рисунках \ref{img:g_10_pca} и \ref{img:g_2_pca_w} представлены примеры решения задачи кластеризации методом Гат-Гевы при различных параметрах.

Ниже в таблицах \ref{tab:clustering} и \ref{tab:clustering1} представлены результаты решения задачи кластеризации разными методами при различном количестве кластеров К.



% \newpage


\includeimage
    {g_7} % Имя файла без расширения (файл должен быть расположен в директории inc/img/)
    {f} % Обтекание (без обтекания)
    {h} % Положение рисунка (см. figure из пакета float)
    {1\textwidth} % Ширина рисунка
    {Результат кластеризации методом Гат-Гевы при стандартный размерностях векторов, при К=7} % Подпись рисунка

\includeimage
    {g_7_pca} % Имя файла без расширения (файл должен быть расположен в директории inc/img/)
    {f} % Обтекание (без обтекания)
    {h} % Положение рисунка (см. figure из пакета float)
    {1\textwidth} % Ширина рисунка
    {Результат кластеризации методом Гат-Гевы при преобразованных векторах и лемматизации, при К=7} % Подпись рисунка

% \newpage

\includeimage
    {g_10_pca} % Имя файла без расширения (файл должен быть расположен в директории inc/img/)
    {f} % Обтекание (без обтекания)
    {h} % Положение рисунка (см. figure из пакета float)
    {1\textwidth} % Ширина рисунка
    {Результат кластеризации методом Гат-Гевы при преобразованных векторах и лемматизации, при К=10} % Подпись рисунка

\includeimage
    {g_2_pca} % Имя файла без расширения (файл должен быть расположен в директории inc/img/)
    {f} % Обтекание (без обтекания)
    {h} % Положение рисунка (см. figure из пакета float)
    {1\textwidth} % Ширина рисунка
    {Результат кластеризации методом Гат-Гевы при преобразованных векторах и лемматизации, при К=2} % Подпись рисунка

% \newpage
 
\includeimage
    {g_7_pca_w} % Имя файла без расширения (файл должен быть расположен в директории inc/img/)
    {f} % Обтекание (без обтекания)
    {h} % Положение рисунка (см. figure из пакета float)
    {1\textwidth} % Ширина рисунка
    {Результат кластеризации методом Гат-Гевы при преобразованных векторах и без лемматизации, при К=10} % Подпись рисунка

\includeimage
    {g_2_pca_w} % Имя файла без расширения (файл должен быть расположен в директории inc/img/)
    {f} % Обтекание (без обтекания)
    {h} % Положение рисунка (см. figure из пакета float)
    {1\textwidth} % Ширина рисунка
    {Результат кластеризации методом Гат-Гевы при преобразованных векторах и без лемматизации, при К=2} % Подпись рисунка

\newpage

% \section{Сравнительная таблица}

% \includelistingpretty
%     {out.txt} % Имя файла с расширением (файл должен быть расположен в директории inc/lst/)
%     {} % Язык программирования (необязательный аргумент)
%     {Пример работы с консольной программой} % Подпись листинга


\begin{table}[h]
\caption{Сравнение результатов кластеризации при лемматизации и различных количествах кластеров К}
\label{tab:clustering}
\centering
\begin{tabular}{>{\centering\arraybackslash}m{6.1cm}| >{\centering\arraybackslash}m{0.5cm} |>{\centering\arraybackslash}m{2.8cm}>{\centering\arraybackslash}m{2.8cm} >{\centering\arraybackslash}m{2.8cm}}
\hline
\textbf{Критерии} & \textbf{K} & \textbf{K-средних} & \textbf{C-средних} & \textbf{Гат-Гева} \\
\hline
Среднее межкластерное расстояние & \multirow{2}{*}[-2ex]{2} & 1,27 & 1,27 & 1,27 \\
\cline{1-1}\cline{3-5}
Среднее внутрикластерное расстояние &  & 1,39 & 1,39 & 1,39 \\
\hline
Среднее межкластерное расстояние & \multirow{2}{*}[-2ex]{3} & 1,27 & 1,27 & 1,27 \\
\cline{1-1}\cline{3-5}
Среднее внутрикластерное расстояние &  & 1,39 & 1,39 & 1,39 \\
\hline
Среднее межкластерное расстояние & \multirow{2}{*}[-2ex]{4} & 1,27 & 1,27 & 1,27 \\
\cline{1-1}\cline{3-5}
Среднее внутрикластерное расстояние &  & 1,39 & 1,39 & 1,39 \\
\hline
Среднее межкластерное расстояние & \multirow{2}{*}[-2ex]{5} & 1,27 & 1,27 & 1,27 \\
\cline{1-1}\cline{3-5}
Среднее внутрикластерное расстояние &  & 1,39 & 1,39 & 1,39 \\
\hline
Среднее межкластерное расстояние & \multirow{2}{*}[-2ex]{6} & 1,27 & 1,27 & 1,27 \\
\cline{1-1}\cline{3-5}
Среднее внутрикластерное расстояние &  & 1,39 & 1,39 & 1,39 \\
\hline
Среднее межкластерное расстояние & \multirow{2}{*}[-2ex]{7} & 1,27 & 1,27 & 1,27 \\
\cline{1-1}\cline{3-5}
Среднее внутрикластерное расстояние &  & 1,39 & 1,39 & 1,39 \\
\hline
Среднее межкластерное расстояние & \multirow{2}{*}[-2ex]{10} & 1,27 & 1,27 & 1,27 \\
\cline{1-1}\cline{3-5}
Среднее внутрикластерное расстояние &  & 1,39 & 1,39 & 1,39 \\
\hline
\end{tabular}
\end{table}

\begin{table}[h]
\caption{Сравнение результатов кластеризации без лемматизации и различных количествах кластеров К}
\label{tab:clustering1}
\centering
\begin{tabular}{>{\centering\arraybackslash}m{6.1cm}| >{\centering\arraybackslash}m{0.5cm} |>{\centering\arraybackslash}m{2.8cm}>{\centering\arraybackslash}m{2.8cm} >{\centering\arraybackslash}m{2.8cm}}
\hline
\textbf{Критерии} & \textbf{K} & \textbf{K-средних} & \textbf{C-средних} & \textbf{Гат-Гева} \\
\hline
Среднее межкластерное расстояние & \multirow{2}{*}[-2ex]{2} & 1,27 & 1,27 & 1,27 \\
\cline{1-1}\cline{3-5}
Среднее внутрикластерное расстояние &  & 1,39 & 1,39 & 1,39 \\
\hline
Среднее межкластерное расстояние & \multirow{2}{*}[-2ex]{3} & 1,27 & 1,27 & 1,27 \\
\cline{1-1}\cline{3-5}
Среднее внутрикластерное расстояние &  & 1,39 & 1,39 & 1,39 \\
\hline
Среднее межкластерное расстояние & \multirow{2}{*}[-2ex]{4} & 1,27 & 1,27 & 1,27 \\
\cline{1-1}\cline{3-5}
Среднее внутрикластерное расстояние &  & 1,39 & 1,39 & 1,39 \\
\hline
Среднее межкластерное расстояние & \multirow{2}{*}[-2ex]{5} & 1,27 & 1,27 & 1,27 \\
\cline{1-1}\cline{3-5}
Среднее внутрикластерное расстояние &  & 1,39 & 1,39 & 1,39 \\
\hline
Среднее межкластерное расстояние & \multirow{2}{*}[-2ex]{6} & 1,27 & 1,27 & 1,27 \\
\cline{1-1}\cline{3-5}
Среднее внутрикластерное расстояние &  & 1,39 & 1,39 & 1,39 \\
\hline
Среднее межкластерное расстояние & \multirow{2}{*}[-2ex]{7} & 1,27 & 1,27 & 1,27 \\
\cline{1-1}\cline{3-5}
Среднее внутрикластерное расстояние &  & 1,39 & 1,39 & 1,39 \\
\hline
Среднее межкластерное расстояние & \multirow{2}{*}[-2ex]{10} & 1,27 & 1,27 & 1,27 \\
\cline{1-1}\cline{3-5}
Среднее внутрикластерное расстояние &  & 1,39 & 1,39 & 1,39 \\
\hline
\end{tabular}
\end{table}

\chapter*{ЗАКЛЮЧЕНИЕ}

В ходе выполнения лабораторной работы была решена задача кластеризации текстовых документов с использованием трёх методов: K-средних, C-средних и Гат-Гевы.

Основные выводы:

\begin{enumerate}
    \item Предобработка данных и векторизация являются критически важными этапами. Применение лемматизации позволило снизить размерность пространства признаков за счёт приведения слов к их начальным формам, что улучшило качество кластеризации;
    \item Метод K-средних продемонстрировал стабильность и предсказуемость результатов как при использовании лемматизации, так и без неё;
    \item Метод C-средних показал менее стабильные результаты -- результаты кластеризации при некоторых количествах кластеров были непредсказуемы. К нестабильности работы стоит также добавить требование настройки параметра нечёткости \( m \);
    \item Проведенные расчеты выявили существование неочевидного порога количества кластеров \(К=5\) при реализации метода Гат-Гевы. При значениях \(К>5\) процедура кластеризации завершается без ошибок, но не производит необходимого разбиения. Неясно, является ли данное ограничение свойством самого алгоритма или особенностью конкретной программной реализации;
    \item Сравнение с экспертной разметкой и анализ метрик (среднее внутрикластерное и межкластерное расстояние) позволили оценить, насколько автоматически полученные кластеры соответствуют тематической структуре данных. Наилучшие результаты были достигнуты при числе кластеров, равном к экспертному \( K = 7 \).
\end{enumerate}

